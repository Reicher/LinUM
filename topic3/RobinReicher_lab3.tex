%----------------------------------------------------------------------------------------
%	PACKAGES AND OTHER DOCUMENT CONFIGURATIONS
%----------------------------------------------------------------------------------------

\documentclass[11pt]{article}

\usepackage[utf8]{inputenc} % Required for inputting international characters
\usepackage[T1]{fontenc} % Output font encoding for international characters
\usepackage{lipsum} % Used for inserting dummy 'Lorem ipsum' text into the template
\usepackage{pdflscape}
\usepackage{longtable}
\usepackage{enumitem}
\usepackage{listings}             % Include the listings-package
\usepackage{scrextend}
\usepackage{mathpazo} % Palatino font

\begin{document}


%----------------------------------------------------------------------------------------
%	TITLE PAGE
%----------------------------------------------------------------------------------------

\begin{titlepage} % Suppresses displaying the page number on the title page and the subsequent page counts as page 1
	\newcommand{\HRule}{\rule{\linewidth}{0.5mm}} % Defines a new command for horizontal lines, change thickness here

	\center % Centre everything on the page

	%------------------------------------------------
	%	Headings
	%------------------------------------------------

	\textsc{\LARGE Umeå Universitet}\\[1.5cm] % Main heading such as the name of your university/college

	\textsc{\Large Linux som utvecklingsmiljö}\\[0.5cm] % Major heading such as course name

	\textsc{\large HT-17}\\[0.5cm] % Minor heading such as course title

	%------------------------------------------------
	%	Title
	%------------------------------------------------

	\HRule\\[0.4cm]

	{\huge\bfseries Övning 3. Versionshantering}\\[0.4cm] % Title of your document

	\HRule\\[1.5cm]

	%------------------------------------------------
	%	Author(s)
	%------------------------------------------------

	\begin{minipage}{0.4\textwidth}
		\begin{flushleft}
			\large
			Robin Reicher
		\end{flushleft}
	\end{minipage}
	~
	\begin{minipage}{0.4\textwidth}
		\begin{flushright}
			\large
			850316-6657
		\end{flushright}
	\end{minipage}

	% If you don't want a supervisor, uncomment the two lines below and comment the code above
	%{\large\textit{Author}}\\
	%John \textsc{Smith} % Your name

	%------------------------------------------------
	%	Date
	%------------------------------------------------

	\vfill\vfill\vfill % Position the date 3/4 down the remaining page

	{\large\today} % Date, change the \today to a set date if you want to be precise

	%------------------------------------------------
	%	Logo
	%------------------------------------------------

	%\vfill\vfill
	%\includegraphics[width=0.2\textwidth]{placeholder.jpg}\\[1cm] % Include a department/university logo - this will require the graphicx package

	%----------------------------------------------------------------------------------------

	\vfill % Push the date up 1/4 of the remaining page

\end{titlepage}

%----------------------------------------------------------------------------------------

\section{CVS}


\begin{enumerate}
\item Laddar ner och hämtar hem cvs med apt-get. \\
sudo apt-get install cvs
\item Skapar en mapp för repot och sätter variablen CVSROOT till denna.\\
cd /home/regen/Documents/LinUM/topic3/ \\
mkdir repo \\
export CVSROOT=:local:/home/regen/Documents/LinUM/topic3/repo
\item Kör cvs init i repot\\
cd repo\\
cvs init
\item Skapa en tom map som representerar alla project \\
cd ..\\
mkdir tomMapp
\item Går in i mappen och importera allt `innehåll' till repot. Sätter Robin som `vendor-tag' och R1 som `release-tag'. Efteråt import tar jag bort mappen.\\
cd tomMapp\\
cvs import LinUM Robin R1\\
cd ..\\
rmdir tomMapp
\item Skapar en `arbetsmapp' och `checkar ut' projekt dit.\\
mkdir work\\
cd work\\
cvs co LinUM
\item Skapar en mapp och lägger till och checkar in den.\\
cd LinUm\\
mkdir Topic3\\
cvs add Topic3\\
cvs ci\\
\item Skapar två filer, lägger till dem och checka in efteråt. \\
cd Topic3/\\
echo "Horse" > EnFil.txt\\
echo "Chicken" > EnAnnanFil.txt\\
cvs add EnFil.txt \\
cvs add EnAnnanFil.txt \\
cvs co
\item Ändrar i en fil och checka in\\
echo "Ducks" > EnFil.txt
\end{enumerate}

I repot ligger nu EnFil.txt med innehållet: 
\begin{verbatim}
head	1.2;
access;
symbols;
locks; strict;
comment	@# @;


1.2
date	2017.09.14.17.29.22;	author regen;	state Exp;
branches;
next	1.1;
commitid	10059BABC5803E46A57;

1.1
date	2017.09.14.17.26.36;	author regen;	state Exp;
branches;
next	;
commitid	10059BABBB201FD653C;


desc
@@


1.2
log
@Changed Horse to ducks, more efficient.
@
text
@Ducks
@


1.1
log
@FIrst commit
@
text
@d1 1
a1 1
Horse
@

\end{verbatim}

I översta halvan ser man en kort sammanfattning av filen och de olika (2) versioner som finns för filen. I undra halvan syns alla de ändringar som gjorts för denna fil. 

\newpage

\section{Subversion}
\begin{enumerate}
\item Installerar svn med apt-get \\
sudo apt-get install subversion
\item Skapar en arbetsmapp och checkar ut projektet\\
mkdir svn\_work \\
cd svn\_work \\
svn co svn://130.239.163.12/labb4 . --username labb4
\item Skapar min personliga mapp och fil, lägger till mappen och commitar ändringarna. \\
mkdir robin\_reicher \\
echo "Hello svn world" > robin\_reicher/bl.txt \\
svn add robin\_reicher/ \\
svn commit -m "Add personal folder"
\item Lägger till mitt namn till users-mappen, commitar sedan. \\
echo "Robin Reicher" >> users \\
svn commit -m "Add user Robin Reicher to user-file"
\item Lekte runt lite med kommandon...
\item Gör ytterligare en kopia av repot \\
mkdir ../svn\_work\_2 \\
cd ../svn\_work\_2 \\
svn co svn://130.239.163.12/labb4 . --username labb4
\item Ändrar en hel del i min personliga fil, commitar utan problem. \\
echo "Hello cruel svn world" > robin\_reicher/bl.txt \\
echo "Goodbye cruel svn world" >> robin\_reicher/bl.txt \\ 
svn commit -m "Change and add line in bl"
\item Ändrar en hel del i det andra repot och försöker commita, vilket går dåligt. \\
cd ../svn\_work \\
echo "Hello cool svn world" > robin\_reicher/bl.txt \\
echo "How are you?" >> robin\_reicher/bl.txt \\
echo "Goodbye cool svn world" >> robin\_reicher/bl.txt \\
svn commit -m "added new lines and changed some" \\
\item Försöker istället uppdatera repot, får upp information där jag kan välja att kolla diffen med mera. Efter en titt på diffen (df) tar jag postpone(p). 
svn update
\begin{verbatim}
Updating '.':
C    robin\_reicher/bl.txt
Updated to revision 2724.
Summary of conflicts:
  Text conflicts: 1
Conflict discovered in file 'robin\_reicher/bl.txt'.
Select: (p) postpone, (df) show diff, (e) edit file, (m) merge,
        (mc) my side of conflict, (tc) their side of conflict,
        (s) show all options: df
\end{verbatim}
\item Alla ändringar ligger nu i min lokala fil med kommentarer om vilka rader som kommer varifrån. Öppnar filen i en text-editor\\
emacs robin\_reicher/bl.txt  \\
\begin{verbatim}
<<<<<<< .mine
Hello cool svn world
How are you?
Goodbye cool svn world
||||||| .r2723
Hello cool svn world
Nice to meet you!
Goodbye cool svn world
=======
Hello cruel svn world
Goodbye cruel svn world
>>>>>>> .r2724
\end{verbatim}
\item Redigerar filen som jag vill ha den och sparar.
\begin{verbatim}
Hello cool svn world
Nice to meet you!
How are you?
Goodbye cruel svn world
\end{verbatim}
\item Berättar för svn att de konflikterna filen hade nu är lösta. \\
svn resolve --accept=working robin\_reicher/bl.txt
\item Commitar nu utan problem. \\
svn commit -m "add new lines and resolve conflicts" (lyckas)
\end{enumerate}

\end{document}
