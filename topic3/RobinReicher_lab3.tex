%----------------------------------------------------------------------------------------
%	PACKAGES AND OTHER DOCUMENT CONFIGURATIONS
%----------------------------------------------------------------------------------------

\documentclass[11pt]{article}

\usepackage[utf8]{inputenc} % Required for inputting international characters
\usepackage[T1]{fontenc} % Output font encoding for international characters
\usepackage{lipsum} % Used for inserting dummy 'Lorem ipsum' text into the template
\usepackage{pdflscape}
\usepackage{longtable}
\usepackage{enumitem}
\usepackage{listings}             % Include the listings-package
\usepackage{scrextend}
\usepackage{mathpazo} % Palatino font

\begin{document}


%----------------------------------------------------------------------------------------
%	TITLE PAGE
%----------------------------------------------------------------------------------------

\begin{titlepage} % Suppresses displaying the page number on the title page and the subsequent page counts as page 1
	\newcommand{\HRule}{\rule{\linewidth}{0.5mm}} % Defines a new command for horizontal lines, change thickness here

	\center % Centre everything on the page

	%------------------------------------------------
	%	Headings
	%------------------------------------------------

	\textsc{\LARGE Umeå Universitet}\\[1.5cm] % Main heading such as the name of your university/college

	\textsc{\Large Linux som utvecklingsmiljö}\\[0.5cm] % Major heading such as course name

	\textsc{\large HT-17}\\[0.5cm] % Minor heading such as course title

	%------------------------------------------------
	%	Title
	%------------------------------------------------

	\HRule\\[0.4cm]

	{\huge\bfseries Övning 3. Versionshantering}\\[0.4cm] % Title of your document

	\HRule\\[1.5cm]

	%------------------------------------------------
	%	Author(s)
	%------------------------------------------------

	\begin{minipage}{0.4\textwidth}
		\begin{flushleft}
			\large
			Robin Reicher
		\end{flushleft}
	\end{minipage}
	~
	\begin{minipage}{0.4\textwidth}
		\begin{flushright}
			\large
			850316-6657
		\end{flushright}
	\end{minipage}

	% If you don't want a supervisor, uncomment the two lines below and comment the code above
	%{\large\textit{Author}}\\
	%John \textsc{Smith} % Your name

	%------------------------------------------------
	%	Date
	%------------------------------------------------

	\vfill\vfill\vfill % Position the date 3/4 down the remaining page

	{\large\today} % Date, change the \today to a set date if you want to be precise

	%------------------------------------------------
	%	Logo
	%------------------------------------------------

	%\vfill\vfill
	%\includegraphics[width=0.2\textwidth]{placeholder.jpg}\\[1cm] % Include a department/university logo - this will require the graphicx package

	%----------------------------------------------------------------------------------------

	\vfill % Push the date up 1/4 of the remaining page

\end{titlepage}

%----------------------------------------------------------------------------------------

\section{CVS}
1. Välj en plats för repot och sätt variablen CVSROOT till denna.
export CVSROOT=:local:/home/regen/Documents/LinUM/topic3/repo

2. Skapa en mapp på platsen och kör cvs init.
mkdir /home/regen/Documents/LinUM/topic3/repo
cd /home/regen/Documents/LinUM/topic3/repo
cvs init

3. Skapa en tom map som representerar dina project
mkdir /home/regen/Documents/LinUM/topic3/tomMapp

4. Gå in i mappen och importera allt `innehåll' till ditt repo. Välj ett passande namn och en `vendor-tag' och `release-tag', efteråt kan man ta bort mappen.
cd /home/regen/Documents/LinUM/topic3/tomMapp
cvs import LinUm Robin R1
cd ..
rmdir tomMapp

5. Skapa en `arbetsmapp' och `checka ut' dina projekt dit.
mkdir /home/regen/Documents/LinUM/topic3/work
cd /home/regen/Documents/LinUM/topic3/work
cvs co LinUm

6. Skapa ett projekt och lägg till det och checka in det.
cd /home/regen/Documents/LinUM/topic3/work/LinUm/
mkdir Topic3
cvs add Topic3
cvs ci

\section{Subversion}
\lipsum[3]

\end{document}
