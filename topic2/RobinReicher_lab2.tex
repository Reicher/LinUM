%----------------------------------------------------------------------------------------
%	PACKAGES AND OTHER DOCUMENT CONFIGURATIONS
%----------------------------------------------------------------------------------------

\documentclass[11pt]{article}

\usepackage[utf8]{inputenc} % Required for inputting international characters
\usepackage[T1]{fontenc} % Output font encoding for international characters
\usepackage{lipsum} % Used for inserting dummy 'Lorem ipsum' text into the template
\usepackage{pdflscape}
\usepackage{longtable}
\usepackage{enumitem}
\usepackage{listings}             % Include the listings-package
\usepackage{scrextend}
\usepackage{mathpazo} % Palatino font

\begin{document}


%----------------------------------------------------------------------------------------
%	TITLE PAGE
%----------------------------------------------------------------------------------------

\begin{titlepage} % Suppresses displaying the page number on the title page and the subsequent page counts as page 1
	\newcommand{\HRule}{\rule{\linewidth}{0.5mm}} % Defines a new command for horizontal lines, change thickness here

	\center % Centre everything on the page

	%------------------------------------------------
	%	Headings
	%------------------------------------------------

	\textsc{\LARGE Umeå Universitet}\\[1.5cm] % Main heading such as the name of your university/college

	\textsc{\Large Linux som utvecklingsmiljö}\\[0.5cm] % Major heading such as course name

	\textsc{\large HT-17}\\[0.5cm] % Minor heading such as course title

	%------------------------------------------------
	%	Title
	%------------------------------------------------

	\HRule\\[0.4cm]

	{\huge\bfseries Övning 2. Att arbeta i ett Linux-system}\\[0.4cm] % Title of your document

	\HRule\\[1.5cm]

	%------------------------------------------------
	%	Author(s)
	%------------------------------------------------

	\begin{minipage}{0.4\textwidth}
		\begin{flushleft}
			\large
			Robin Reicher
		\end{flushleft}
	\end{minipage}
	~
	\begin{minipage}{0.4\textwidth}
		\begin{flushright}
			\large
			850316-6657
		\end{flushright}
	\end{minipage}

	% If you don't want a supervisor, uncomment the two lines below and comment the code above
	%{\large\textit{Author}}\\
	%John \textsc{Smith} % Your name

	%------------------------------------------------
	%	Date
	%------------------------------------------------

	\vfill\vfill\vfill % Position the date 3/4 down the remaining page

	{\large\today} % Date, change the \today to a set date if you want to be precise

	%------------------------------------------------
	%	Logo
	%------------------------------------------------

	%\vfill\vfill
	%\includegraphics[width=0.2\textwidth]{placeholder.jpg}\\[1cm] % Include a department/university logo - this will require the graphicx package

	%----------------------------------------------------------------------------------------

	\vfill % Push the date up 1/4 of the remaining page

\end{titlepage}

%----------------------------------------------------------------------------------------

\section{Beskrivningar av samtliga kommandon i del 1}
Nedan listas alla kommandon i del 1 samt de eventuella växlar en användare kan tänkas använda. 
\begin{labeling}{whereis12}
\item [man] Visar manualen för ett kommando. 
\item [info] Liknande 'man' men mer som en konsol-wiki med länkade sidor. 	
\item [cp] Kopierar filer från en plats till en annan, tar minst 2 argument <källa> och <mål>. '-r' kopierar mappar rekursivt. '-s' skapar en symbolisk länk istället för att kopiera.  
\item [mv] Flyttar eller döper om en fil, tar liksom 'cp' två argument <källa> och <mål>. Om dessa är i samma mapp döps filen om. '-n' skriver inte över existerande filer. 
\item [mkdir] Skapar en mapp. '-m' används för att sätta filens rättigheter, dvs som 'chmod'
\item [rmdir] Raderar en mapp 	
\item [rm] Raderar en fil. '-r' kan radera mappar och deras innehåll rekursivt.
\item [find] Låter dig söka igenom mappstrukturer efter filer baserat på bla. namn.	'-L' söker även igenom symboliska länkar.'-mount ' söker inte igenom andra filsystem.
\item [cd] Öppnar upp en mapp, '..' som argument "backar ur" din aktiva mapp 	
\item [pwd]  Visar var du befinner dig i mappstrukturen. '-P' visar fysisk plats och undviker alla symboliska länkar. 
\item [df] Visar information om filsystemet och diskanvändning. '-a' visar alla filsystem, även onåbara, dubbletter m.m.
\item [ps] Listar de processer som för tillfället körs. '-elf' visar info om trådar. 
\item [du] Listar alla mappars storlek, men en fil som argument visas den filens storlek. '-a' visar alla filer och mappars storlekar. 
\item [tar] Komprimera eller packa upp filer. För att komprimera kan man använda '-cf' följt av den nya filens namn och sedan en lista på de filer/mappars om ska ingå. För att packa upp används  '-xf' följt av arkivets namn. 	
\item [seq] Skriver ut en sekvens av nummer. Kan ha 1-3 argument. Ett argument skapar en sekvens av heltal från 1 till argumentet. Två argument listar alla heltal mellan argumenten. Tre argument låter dig välja starttal, inkrementerings-takt och sluttal.	
\item [whoami] Skriver ut aktiv använder id. 	
\item [users] Listar alla inloggade användare	
\item [who] Listar mer utförlig info om vilka användare som är inloggade. Växeln --boot visar när systemet startades. '-a' visar alla användare. 
\item [whereis] Visar var en binär(-b), källfil(-s) eller manual-sida(-m) finns i filsystemet. 	
\item [cat] Kan läsa filer till standard-output. '-a' visar alla tecken, så som radslut m.m.
\item [tee] Skriver input till en fil och standard output. '-a' skriver inte över filen utan lägger till i slutet på den. 
\item [more] delar upp lång text i hela sidor, börjar på första och låter dig bläddra med <mellanslag>. '-d' ger lite hjälpande förklaringar. 	
\item [less] En nyare variant av 'more' som låter dig bläddra bakåt också.	
\item [uniq] Filtrerar bort rader som inte är unika. '-d' skriver istället bara ut icke unika rader. Med 'f' kan man låta bli att kontrollera några rader i början.  	
\item [tail] Visar de sista 10 raderna i en fil eller input. Med '-n' kan man ändra till att visa fler eller färre. 
\item [echo] Visar en rad med text. 	
\item [which] Visar var på filsystemet ett program ligger. 	
\item [wget] Används för att hämta data över ett nätverk.  Med '-q' görs detta utan utskrifter. 
\item [cut] Filtrerar bort delar av raderna i en fil. Med '-c' kan man välja att klippa ur de första tecknen i varje rad eller tex. tecken 5-10. 
\item [grep] Söker efter text i en fil och visar hela raden. '-i' ignorerar skillnader mellan stora och små bokstäver. '-r' söker rekursivt igenom en mapp. 
\item [sort] Sorterar indata, -R slumpar den istället. '-b' ignorerar inledande blanksteg.	
\item [wc] Räknar antal rader i en fil. Kan även räkna på ord(-w), bytes(-c) med mera.
\end{labeling}

\section{I uppgift 2 står det att ingen annan användare ska ha tillgång till katalogerna. Är detta möjligt eller finns det en användare som aldrig går att utestänga?}
Det är omöjligt, man kan aldrig utestänga användaren 'root'. Man kan alltid logga in som root (sudo -i) och komma åt allt. 

\pagebreak

\section{Hur du löste uppgift 2, och bifoga kopior på adekvata delar från filerna /etc/group, /etc/passwd samt utskrifter från ls -l}

\begin{lstlisting}[language=bash,caption={Utdrag från /etc/passwd}]
$ tail -n 3 /etc/passwd
adaand:x:1002:1002::/home/adaand:/bin/bash
petpet:x:1003:1003::/home/petpet:/bin/bash
lisper:x:1004:1004:,,,:/home/lisper:/bin/bash
\end{lstlisting}


\begin{lstlisting}[language=bash,caption={Utdrag från /etc/group}]
$ tail -n 3 /etc/group
datagroup:x:1005:adaand,petpet
admingroup:x:1006:adaand,lisper
marketgroup:x:1007:lisper,petpet
\end{lstlisting}

\begin{lstlisting}[language=bash,caption={Resultat av ls -l}]
$ ls -l
total 20
d---rwx--- 2 regen admingroup  4096 sep  9 13:58 admin
d---rwx--- 2 regen datagroup   4096 sep  9 13:58 data
d---rwx--- 2 regen marketgroup 4096 sep  9 13:58 market
\end{lstlisting}

Användare 'regen' är min användare som jag vanligtvis använder. De olika grupper har fullständiga rättigheter till mapparna. Varken ägaren 'regen' eller övriga användare kan varken läsa, skriva eller exekvera.
     
\section{Beskrivningar av katalogerna i uppgift 3}
\begin{labeling}{/stuffs}
\item [/boot] De filer som krävs för att starta ditt system. Linux kernel m.m 
\item [/etc] Generella konfigurationsfiler för hela systemet. Kan ofta läsbara och möjliga att redigera "för hand" med en texteditor.
\item [/sbin] Binärer till administratörsverktyg som i regel används för systemunderhåll av root
\item [/bin] Grundläggande binärer som användare kan tänkas använda. Exempel kan vara bash, cp, ls och ps.
\item [/usr] Innehåller bland annat icke kritiska binärer (/usr/bin och /usr/sbin), bibliotek till dessa (/usr/lib) samt program som installerats från källfiler (/usr/locale)
\item [/var] Var står får "variable files" och innehåller filer som kan tänkas växa. Det kan handla om logg-filer, databaser med mera.
\item [/dev] Innehåller representationer av systemets olika "devices" vilket generellt handlar om hårddiskar, mus, tangentbord, USB och annan hårdvara. Förutom hårvara finns även en slumpgenerator (/dev/random) och det bottenlösa hålet /dev/null
\item [/home] Innehåller en katalog för varje användare på systemet (förutom root). I dessa kataloger ligger användarspecifika inställningar och data.
\end{labeling}
    
\section{Svaren på del 4 med kommentarer om hur varje del i kombinationerna fungerar}
\begin{enumerate}[label=(\alph*)]
\item find -size  +1000 | sort \\ 
Hittar alla filer med size över 1000 kilobytes och pipar det till sort för sortering
\item sort adress.txt | uniq | wc -l \\
Först sorterar vi adress.txt, sen plockar vi ut de unika rader som finns, sist använder vi wc med växeln -l för att räkna antal rader. 
\item users | sort | uniq
'users' listar alla användare, vi pipar det till sort som sorterar för att slutligen pipa till uniq som tar bort eventuella dubbletter.
\item seq 1 10 | tee test1 test2
seq 1 10 listar alla heltal mellan 1 och 10, vi pipar det till tee som skriver till både filer och standard output.
\end{enumerate}


\section{Redovisa ingående vilka slutsatser du drar av resultaten från de olika kommandona i del 5. Hur interagerar programmen med de olika fildeskriptorerna och hur styr man flödena? Redovisa även svaren på frågorna.}
\begin{labeling}{ls /hh 2> lserror2.txttxt}
\item [ls / > lsoutput.txt]
Ingen utskrift, lsoutput.txt innehåller resultatet. '>' (egentligen 1>) skriver STDOUT till filen istället för skärmen

\item [ls /hh > lsoutput2.txt]
Får "ls: cannot access '/hh': No such file or directory" som utskrift. lsoutput2.txt är tom. Felmeddelandet hamnar inte i filen eftersom det skrivs till STDERROR (vilket också visas på skärmen).

\item [ls / 2> lserror.txt]
Resultatet av 'ls /' visas på skärmen, eventuella felmeddelanden hade skrivits till lserror.txt efter som '2>' användes. Men nu gick allt fint och filen är tom.

\item [ls /hh 2> lserror2.txt]
Ingen utskrift ges på skärmen eftersom STDOUT inte har något att skriva ut och det felmeddelande som gavs har skrivits till lserror2.txt med hjälp av 2>.
\end{labeling}

De tre fildescriptorerna (STDIN, STDOUT och STDERROR) används för att styra flöden in och ut från filer. STDIN är vanligen satt till tangentbordet i konsolen. Detta flöde kan ändras genom '0<', exempelvis kan styra flödet från en fil, tex:\\*
cat 0< enFilMedFilnamn.txt\\
STDOUT är vanligen satt till skärmen i konsolen. För att spara dessa utskrifter till tex. en fil kan man styra om flödet med '1>' eller bara '>'. Tex: \\*
ps > enFil.txt\\
STDERROR skrivs precis som STDOUT ut till skärmen i konsolen. Till skillnad från STDOUT är denna vanligvis reserverad för felutskrifter. För att styra om eventuella felutskrifter använder man '2>' på samma sätt som med STDOUT. \\*
Vanligvis när man använder '>' så skrivs filens innehåll över. Genom att använda '>>' lägger man istället till text i slutet av filen. 
\end{document}
