%----------------------------------------------------------------------------------------
%	PACKAGES AND OTHER DOCUMENT CONFIGURATIONS
%----------------------------------------------------------------------------------------

\documentclass[11pt]{article}

\usepackage[utf8]{inputenc} % Required for inputting international characters
\usepackage[T1]{fontenc} % Output font encoding for international characters
\usepackage{lipsum} % Used for inserting dummy 'Lorem ipsum' text into the template
\usepackage{pdflscape}
\usepackage{longtable}
\usepackage{enumitem}
\usepackage{listings}             % Include the listings-package
\usepackage{scrextend}
\usepackage{mathpazo} % Palatino font

\begin{document}

\renewcommand{\thesubsection}{\thesection.\alph{subsection}}

%----------------------------------------------------------------------------------------
%	TITLE PAGE
%----------------------------------------------------------------------------------------

\begin{titlepage} % Suppresses displaying the page number on the title page and the subsequent page counts as page 1
	\newcommand{\HRule}{\rule{\linewidth}{0.5mm}} % Defines a new command for horizontal lines, change thickness here

	\center % Centre everything on the page

	%------------------------------------------------
	%	Headings
	%------------------------------------------------

	\textsc{\LARGE Umeå Universitet}\\[1.5cm] % Main heading such as the name of your university/college

	\textsc{\Large Linux som utvecklingsmiljö}\\[0.5cm] % Major heading such as course name

	\textsc{\large HT-17}\\[0.5cm] % Minor heading such as course title

	%------------------------------------------------
	%	Title
	%------------------------------------------------

	\HRule\\[0.4cm]

	{\huge\bfseries Övning 7. Olika verktyg}\\[0.4cm] % Title of your document

	\HRule\\[1.5cm]

	%------------------------------------------------
	%	Author(s)
	%------------------------------------------------

	\begin{minipage}{0.4\textwidth}
		\begin{flushleft}
			\large
			Robin Reicher
		\end{flushleft}
	\end{minipage}
	~
	\begin{minipage}{0.4\textwidth}
		\begin{flushright}
			\large
			850316-6657
		\end{flushright}
	\end{minipage}

	% If you don't want a supervisor, uncomment the two lines below and comment the code above
	%{\large\textit{Author}}\\
	%John \textsc{Smith} % Your name

	%------------------------------------------------
	%	Date
	%------------------------------------------------

	\vfill\vfill\vfill % Position the date 3/4 down the remaining page

	{\large\today} % Date, change the \today to a set date if you want to be precise

	%------------------------------------------------
	%	Logo
	%------------------------------------------------

	%\vfill\vfill
	%\includegraphics[width=0.2\textwidth]{placeholder.jpg}\\[1cm] % Include a department/university logo - this will require the graphicx package

	%----------------------------------------------------------------------------------------

	\vfill % Push the date up 1/4 of the remaining page

\end{titlepage}

%----------------------------------------------------------------------------------------

\section{Man-sidor och Groff}

\section{Diff och patch}

\subsection{  }
Jag skapar de båda filerna fil1 och fil2, eller file1 och file2 som dom kallas högst upp i uppgiften. Med hjälp av kommandot:\\ 
\$ diff fil1 fil2 > diffile\\
skapar jag diff-filen "diffile". Skapar sedan kopior av fil 1 och 2. Slutligen applicerar jag diff-filen på fil1: \\
\$ patch fil1 diffile\\
Efter detta är fil1 identisk med fil2. 
\subsection{}
\subsubsection{Vad betyder de olika argumenten till diff och patch?}
r - "recursive" Betyder att diffen även kommer göras på alla underkataloger och deras underkataloger osv.\\
u - "unified" för diff-filen mer lättläslig för en människa och visar även lite mer information. Man kan se tidpunkten för förändringarna samt lite kontext runt förändringen. \\
N - "NewFile" När man gör diffar mellan strukturer kan det hända att hela filer saknas. Med -N låtsas man då att de saknade filerna existerar men är tomma. \\

\subsubsection{Beskriv tillvägagångssättet för att göra detta}



\section{ Awk och sed}
\subsection{awk}Följande kommando listar alla filer och eventuella kommentarer\\
\$ awk -F: '{ print "Användare: " $1 ", kommentar: " $5  }' /etc/passwd

\subsection{sed}
Följande kommando läser upp filen 3b.txt, pipar det till ett sed-kommando som byter ut å mot 'aa' globalt, sedans skickas det vidare till ett liknande sed-kommando i 2 led som byter ut ä och ö:\\
\$ cat 3b.txt | sed s/å/aa/g | sed s/ä/ae/g | sed s/ö/oe/g > 3b\_fixed.txt


\end{document}
