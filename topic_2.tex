%----------------------------------------------------------------------------------------
%	PACKAGES AND OTHER DOCUMENT CONFIGURATIONS
%----------------------------------------------------------------------------------------

\documentclass[11pt]{article}

\usepackage[utf8]{inputenc} % Required for inputting international characters
\usepackage[T1]{fontenc} % Output font encoding for international characters
\usepackage{lipsum} % Used for inserting dummy 'Lorem ipsum' text into the template
\usepackage{pdflscape}

\usepackage{mathpazo} % Palatino font

\begin{document}


%----------------------------------------------------------------------------------------
%	TITLE PAGE
%----------------------------------------------------------------------------------------

\begin{titlepage} % Suppresses displaying the page number on the title page and the subsequent page counts as page 1
	\newcommand{\HRule}{\rule{\linewidth}{0.5mm}} % Defines a new command for horizontal lines, change thickness here

	\center % Centre everything on the page

	%------------------------------------------------
	%	Headings
	%------------------------------------------------

	\textsc{\LARGE Umeå Universitet}\\[1.5cm] % Main heading such as the name of your university/college

	\textsc{\Large Linux som utvecklingsmiljö}\\[0.5cm] % Major heading such as course name

	\textsc{\large HT-17}\\[0.5cm] % Minor heading such as course title

	%------------------------------------------------
	%	Title
	%------------------------------------------------

	\HRule\\[0.4cm]

	{\huge\bfseries Övning 2. Att arbeta i ett Linux-system}\\[0.4cm] % Title of your document

	\HRule\\[1.5cm]

	%------------------------------------------------
	%	Author(s)
	%------------------------------------------------

	\begin{minipage}{0.4\textwidth}
		\begin{flushleft}
			\large
			Robin Reicher
		\end{flushleft}
	\end{minipage}
	~
	\begin{minipage}{0.4\textwidth}
		\begin{flushright}
			\large
			850316-6657
		\end{flushright}
	\end{minipage}

	% If you don't want a supervisor, uncomment the two lines below and comment the code above
	%{\large\textit{Author}}\\
	%John \textsc{Smith} % Your name

	%------------------------------------------------
	%	Date
	%------------------------------------------------

	\vfill\vfill\vfill % Position the date 3/4 down the remaining page

	{\large\today} % Date, change the \today to a set date if you want to be precise

	%------------------------------------------------
	%	Logo
	%------------------------------------------------

	%\vfill\vfill
	%\includegraphics[width=0.2\textwidth]{placeholder.jpg}\\[1cm] % Include a department/university logo - this will require the graphicx package

	%----------------------------------------------------------------------------------------

	\vfill % Push the date up 1/4 of the remaining page

\end{titlepage}

%----------------------------------------------------------------------------------------

\section{Beskrivningar av samtliga kommandon i del 1}
\begin{center}
    \begin{tabular}{ | l | p{10cm} |}
    \hline
    Kommando & Beskrivning \\ \hline
    man & Visar manualen för ett kommando. Tex. "man ls" \\ \hline
    info & vet inte \\ \hline	
	cp & vet inte \\ \hline	
	mv & vet inte \\ \hline	
	mkdir & vet inte \\ \hline	
	rmdir & vet inte \\ \hline	
	rm & vet inte \\ \hline	
	find & vet inte \\ \hline	
	cd & vet inte \\ \hline	
	pwd & vet inte \\ \hline	
	df & vet inte \\ \hline	
	ps & vet inte \\ \hline	
	du & vet inte \\ \hline	
	tar & vet inte \\ \hline	
	seq & vet inte \\ \hline	
	whoami & vet inte \\ \hline	
	users & vet inte \\ \hline	
	who & vet inte \\ \hline	
	whereis & vet inte \\ \hline	
	cat & vet inte \\ \hline	
	tee & vet inte \\ \hline	
	more & vet inte \\ \hline	
	less & vet inte \\ \hline	
	uniq & vet inte \\ \hline	
	tail & vet inte \\ \hline	
	echo & vet inte \\ \hline	
	which & vet inte \\ \hline	
	wget & vet inte \\ \hline	
	cut & vet inte \\ \hline	
	grep & vet inte \\ \hline	
	sort & vet inte \\ \hline	
	wc & vet inte \\ \hline	    
    \end{tabular}
\end{center}



\section{I uppgift 2 står det att ingen annan användare ska ha tillgång till katalogerna. Är detta möjligt eller finns det en användare som aldrig går att utestänga?}
\lipsum[56]

\section{Hur du löste uppgift 2, och bifoga kopior på adekvata delar från filerna /etc/group, /etc/passwd samt utskrifter från ls -l}
\lipsum[56]
     
\section{Beskrivningar av katalogerna i uppgift 3}
\lipsum[56]

\section{Svaren på del 4 med kommentarer om hur varje del i kombinationerna fungerar}
\lipsum[56]

\section{Redovisa ingående vilka slutsatser du drar av resultaten från de olika kommandona i del 5. Hur interagerar programmen med de olika fildeskriptorerna och hur styr man flödena? Redovisa även svaren på frågorna.}
\lipsum[56]

\end{document}
