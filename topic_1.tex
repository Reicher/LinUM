%%%%%%%%%%%%%%%%%%%%%%%%%%%%%%%%%%%%%%%%%
% Academic Title Page
% LaTeX Template
% Version 2.0 (17/7/17)
%
% This template was downloaded from:
% http://www.LaTeXTemplates.com
%
% Original author:
% WikiBooks (LaTeX - Title Creation) with modifications by:
% Vel (vel@latextemplates.com)
%
% License:
% CC BY-NC-SA 3.0 (http://creativecommons.org/licenses/by-nc-sa/3.0/)
% 
% Instructions for using this template:
% This title page is capable of being compiled as is. This is not useful for 
% including it in another document. To do this, you have two options: 
%
% 1) Copy/paste everything between \begin{document} and \end{document} 
% starting at \begin{titlepage} and paste this into another LaTeX file where you 
% want your title page.
% OR
% 2) Remove everything outside the \begin{titlepage} and \end{titlepage}, rename
% this file and move it to the same directory as the LaTeX file you wish to add it to. 
% Then add \input{./<new filename>.tex} to your LaTeX file where you want your
% title page.
%
%%%%%%%%%%%%%%%%%%%%%%%%%%%%%%%%%%%%%%%%%

%----------------------------------------------------------------------------------------
%	PACKAGES AND OTHER DOCUMENT CONFIGURATIONS
%----------------------------------------------------------------------------------------

\documentclass[11pt]{article}

\usepackage[utf8]{inputenc} % Required for inputting international characters
\usepackage[T1]{fontenc} % Output font encoding for international characters
\usepackage{lipsum} % Used for inserting dummy 'Lorem ipsum' text into the template

\usepackage{pdflscape}
\usepackage{pgfgantt}

\usepackage{mathpazo} % Palatino font

\definecolor{barblue}{RGB}{153,204,254}
\definecolor{groupblue}{RGB}{51,102,254}
\definecolor{linkred}{RGB}{165,0,33}

\begin{document}

%----------------------------------------------------------------------------------------
%	TITLE PAGE
%----------------------------------------------------------------------------------------

\begin{titlepage} % Suppresses displaying the page number on the title page and the subsequent page counts as page 1
	\newcommand{\HRule}{\rule{\linewidth}{0.5mm}} % Defines a new command for horizontal lines, change thickness here
	
	\center % Centre everything on the page
	
	%------------------------------------------------
	%	Headings
	%------------------------------------------------
	
	\textsc{\LARGE Umeå Universitet}\\[1.5cm] % Main heading such as the name of your university/college
	
	\textsc{\Large Linux som utvecklingsmiljö}\\[0.5cm] % Major heading such as course name
	
	\textsc{\large HT-17}\\[0.5cm] % Minor heading such as course title
	
	%------------------------------------------------
	%	Title
	%------------------------------------------------
	
	\HRule\\[0.4cm]
	
	{\huge\bfseries Övning 1. Planera dina studier}\\[0.4cm] % Title of your document
	
	\HRule\\[1.5cm]
	
	%------------------------------------------------
	%	Author(s)
	%------------------------------------------------
	
	\begin{minipage}{0.4\textwidth}
		\begin{flushleft}
			\large
			Robin Reicher
		\end{flushleft}
	\end{minipage}
	~
	\begin{minipage}{0.4\textwidth}
		\begin{flushright}
			\large
			850316-6657
		\end{flushright}
	\end{minipage}
	
	% If you don't want a supervisor, uncomment the two lines below and comment the code above
	%{\large\textit{Author}}\\
	%John \textsc{Smith} % Your name
	
	%------------------------------------------------
	%	Date
	%------------------------------------------------
	
	\vfill\vfill\vfill % Position the date 3/4 down the remaining page
	
	{\large\today} % Date, change the \today to a set date if you want to be precise
	
	%------------------------------------------------
	%	Logo
	%------------------------------------------------
	
	%\vfill\vfill
	%\includegraphics[width=0.2\textwidth]{placeholder.jpg}\\[1cm] % Include a department/university logo - this will require the graphicx package
	 
	%----------------------------------------------------------------------------------------
	
	\vfill % Push the date up 1/4 of the remaining page
	
\end{titlepage}

%----------------------------------------------------------------------------------------

\section{Personlig presentation}

Jag är 32 år gammal och har användt linux för hemmabruk sedan jag började på universitetet för lite mer än tio år sedan. Hemma använder jag datorn mestadels till multimedia, spel och lättare programmerings-projekt. Sedan -11 

\section{Reflektion över kursen krav}

\lipsum[2] % Dummy text

\begin{landscape}

\section{Tidsplan}
    \begin{ganttchart}[x unit=1.4cm, y unit title=.7cm, y unit chart=1.5cm]{8}
    \gantttitle{2012}{3}
    \gantttitle{2013}{5}\\
    \gantttitlelist{50, 51}{1}
    \gantttitle{Christmas}{2}
    \gantttitlelist{2, ..., 5}{1}\\

    \ganttbar[name=b1]{compile ViMuDat}{1}{1.5} \\
    \ganttbar[name=b2,inline=true]{create stubs}{2.5}{4.5} \\
    \ganttbar{implement metadata}{5.5}{5.5}\\
    \ganttmilestone{metadata works}{5.5}\\
    \ganttgroup{Upload}{5}{6.5}\\
    \ganttbar{implement Matlab controler}{4}{2}\\
    \ganttbar{implement file parser}{4}{1.5}\\
    \ganttbar{implement add record}{5.5}{1}\\
    \ganttbar{implement segmentation}{5.5}{1}\\
    \ganttbar{implement registration}{5.5}{1}\\
    \ganttbar{implement classification}{5.5}{1}\\
    \ganttmilestone{implementation finished}{7}\\
    \end{ganttchart}
\end{landscape}

\end{document}
